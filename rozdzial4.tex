\chapter{Wnioski}

Narzędzie jest bardzo pożytecznym oprogramowaniem. Dla niektórych może się to wydawać trochę dziwne, biorąc pod uwagę fakt, że program jest udostępniany na zasadach licencji open-source. W efekcie prac nad tą aplikacją, powstało oprogramowanie słuażce zabezpieczaniu aplikacji webowych poprzez znajdowanie w nich wszelkiego rodzaju luk bezpieczeństwa.\\

Zaletą narzędzia jest z pewnością mnogich dostępnych pluginów oraz udostępnianych przez nie testów, które można przeprowadzić na dowolnej aplikacji webowej Użytkownik otrzymuje około setki pluginów, które można ze sobą łączyć w dowolny sposób. Każde, dowolne ustawienie konfiguracji można zapisać w postaci profilów. Użytkownik oprogramowania W3AF dostaje out-of-the-box kilka prekonfigurowanych profili z których może korzystać. Wśród tych profili znajduje się między innym OWASP\_TOP10 zawierajcy konfigurację pozwalająca na przetestowanie aplikacji pod kątem 10 najczęściej występujących luk w zabezpieczeniu aplikacji. Lista ta została stworzona przez grupę ekspertów na podstawie badań i ankiet przeprowadzonych wśród topowych dostawców treści internetowych. \\

Dodatkową zaletą aplikacji jest łatwość, z jaką możemy przeglądać zgromadzoną przez program wiedzy w trakcie działania. Minusem jest niestety niska zdolnośc aplikacji do agregacji (łączenia) ze sobą różnych rezultatów. Na szczęście program umożliwia eksport wyników do pliku tekstowego. Do największych wad aplikacji należy jej stabilność oraz brak możliwościu uruchomienia trybu graficznego na sytemie operacyjnym OSX.\\

Naszym zdaniem, używana przez nas aplikacja jest ciekawym narzędziem dostarczającym wiedzę o podstawowych lukach bezpieczeństwa w aplikacji. W3AF może nie jest w stanie ostrzec nas przed bardziej wyrafinowanymi próbami dostępu, lecz jest doskonałym narzędziem wskazującym na podstawowe luki bezpieczeństwa, takiej jak XSS lub clickjacking. Z pewnością, gdyby w3af było rozwijane przez większą liczbę programistów, rezultaty oraz stabilność aplikacji uległaby poprawie.\\

Podsumowując, W3AF jest bardzo ciekawym produktem środowiska open-sourcowego. Nie należy z całkowitą ufnością wierzyć wynikom dostarczanych przez programowanie. Rezultaty jednak mogą być dosyć dobrą wskazówką, gdzie należy szukać podatności aplikacji na ataki.