\chapter{Testy}
\label{cha:testy}

Poniżej prezentujemy testy jakie przeprowadzliśmy wraz z ich krótkim opisem oraz niezbędnymi zmianami w kodzie oraz komentarzem odnoszącym się do rezultatów otrzymywanych z w3af.

\section{Click-Jacking}
\begin{enumerable}
\item Opis podatności\\
CLick-Jacking to złośliwa technika nakłaniające użytkowników do klikania w inne rzeczy niż oni sami myślą, że klikają.
Może dojść do potencjalnego ujawnienia poufnych inforamcji lub przejęcia kontroli nad ich komputerem, po kliknięciu na pozornie niewinną stronę. Click-jacking przybiera też formę wbudowanego kodu lub skryptu, który może został wywołany bez wiedzy użytkownika.
\item Zmiany w kodzie\\
Nie były wymagane w przypadku tej podatności.
\item Komentarz\\
W3AF podczas pierwszego uruchomienia, bez zmian w kodzie, wykrył podatność aplikacji Moodle na brak zabezpieczeń przed ata
\end{enumerable}
