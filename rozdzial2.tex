\chapter{Moodle}
\label{cha:moodle}

Moodle to darmowy open-sourcowy program służący do zarządzania kursami uczelnialnymi. System jest napisany w PHP i dytstrybuowany pod licencją GNU. Stworzony z pedagogicznych pobudek, Moodle jest używany do różnorakiej nauki, zdalnej edukacji. Wykorzystując dostosowywalne zarządzanie opcjami, moodle jest używany do tworzenia prywatnych stron oferujących kursy internetowe dla trenerów oraz nauczycieli. Moodle ( akronim od Modular Object-Oriented Dynamic Learing Environment) pozwala na rozszerzanie i dopasowywania środowiska, wykorzystując szeroko dostępne w społeczności pluginy.

Struktura Moodle jest klasyczna dla portalu wymagającego posiadania kont. W Moodle każdy użytkownik posiada osobne konto, do którego przypisana jest jakaś rola. Rolą może by np. administrator, prowadzący kurs albo uczestnik. Żeby stworzyć konto należy się zarejestrować. Moodle jest napisany w całości w PHP, wykorzystuje serwer Apache oraz bazę danych PostgreSQL.
